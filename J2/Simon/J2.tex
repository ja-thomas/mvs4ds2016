\documentclass[a4paper,10pt]{article}
%\documentclass[a4paper,10pt]{scrartcl}
\usepackage{amsmath}
\usepackage[utf8]{inputenc}

\title{}
\author{}
\date{}

\pdfinfo{%
  /Title    (J2 - Homework)
  /Author   (Simon Rieß)
  /Creator  ()
  /Producer ()
  /Subject  ()
  /Keywords ()
}

\begin{document}
\maketitle

\section{Exercise 2.6}
$A= \begin{pmatrix}
9 & -2 \\
-2 & 6 \end{pmatrix}$\\


a) Is A symmetric?\\
$A^T = A= \begin{pmatrix}
9 & -2 \\
-2 & 6 \end{pmatrix}$\\
Yes it is.\\

b) Show that A is positive definite.\\
R: eigen(A)\\
\$values\\
10  5

All eigenvalues are positive $\rightarrow$ A is positive definite\\

\section{Exercise 2.7}
a) Eigenvalues and Eigenvectors\\
\begin{align*}
  \mid A - \lambda I\mid =& \begin{vmatrix} 9-\lambda & -2 \\ -2 & 6-\lambda \end{vmatrix} = 0\\
  & 54 - 15 \lambda + \lambda^2 - 4 = 0\\
  \lambda_1 = 5\ &\ \lambda_2 = 10
 \end{align*}
 
 \begin{align*}
  A - 5 I =  \begin{pmatrix} 4 & -2 \\ -2 & 1 \end{pmatrix} x = & 0\\
  \begin{pmatrix} 1 & -1/2 \\ -2 & 1 \end{pmatrix} x = & 0\\
  \begin{pmatrix} 1 & -1/2 \\ 0 & 0 \end{pmatrix} x = & 0\\
  x_1 = & \begin{pmatrix} -1/2 & -1 \end{pmatrix} \\
  \frac{x_1}{\mid x_1 \mid }  = & \frac{x_1}{ \sqrt{(-1)^2 + (-1/2)^2 } }\\
  = & \frac{x_1}{1.118034} \\
  = & \begin{pmatrix}  -0.4472136 & - 0.8944272 \end{pmatrix}\\
 \end{align*}

  \begin{align*}
  A - 10 I =  \begin{pmatrix} -1 & -2 \\ -2 & -4 \end{pmatrix} x = & 0\\
  \begin{pmatrix} 1 & 2 \\ -2 & -4 \end{pmatrix} x = & 0\\
  \begin{pmatrix} 1 & 2 \\ 0 & 0 \end{pmatrix} x = & 0\\
  x_1 = & \begin{pmatrix} 2 & -1 \end{pmatrix} \\
  \frac{x_1}{\mid x_1 \mid }  = & \frac{x_1}{ \sqrt{2^2 + (-1)^2 } }\\
  = & \frac{x_1}{\sqrt{5}} \\
  = & \begin{pmatrix}  - 0.8944272 & 0.4472136 \end{pmatrix}\\
 \end{align*}
 
 b) Spectral decomposition\\
 \begin{align*}
  A = & \sum_{i=1}^n \lambda_i e_i e_i^T\\
  = & 5 \begin{pmatrix} \frac{-1}{\sqrt{5}} \\ \frac{-2}{\sqrt{5}} \end{pmatrix} \begin{pmatrix} \frac{-1}{\sqrt{5}} & \frac{-2}{\sqrt{5}} \end{pmatrix} + 10 \begin{pmatrix} \frac{-2}{\sqrt{5}} \\ \frac{-1}{\sqrt{5}} \end{pmatrix} \begin{pmatrix} \frac{-2}{\sqrt{5}} & \frac{-1}{\sqrt{5}} \end{pmatrix} \\
  = & 5 \begin{pmatrix} \frac{1}{5} & \frac{2}{5} \\ \frac{2}{5} & \frac{4}{5} \end{pmatrix} + 10 \begin{pmatrix} \frac{4}{5} & \frac{-2}{5} \\ \frac{-2}{5} & \frac{1}{5} \end{pmatrix}\\
  = & \begin{pmatrix} 9 & -2 \\ -2 & 6 \end{pmatrix} \\
  = & A
 \end{align*}
 
 c) Inverse matrix\\
 \begin{align*}
  A^{-1} & = \frac{1}{9 \cdot 6 - (-2) \cdot (-2)} \begin{pmatrix} 6 & 2 \\ 2 & 9 \end{pmatrix}\\
  = & \begin{pmatrix} 0.12 & 0.04 \\ 0.04 & 0.18 \end{pmatrix}
 \end{align*}



\end{document}
